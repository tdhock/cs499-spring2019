\documentclass[12pt]{article}

\usepackage{amssymb}
\usepackage{amsmath}
\usepackage{fullpage}
\usepackage{setspace}
\DeclareMathOperator*{\argmin}{arg\,min}
\DeclareMathOperator*{\argmax}{arg\,max}
\DeclareMathOperator*{\Lik}{Lik}
\DeclareMathOperator*{\Diag}{Diag}
%\doublespacing

\begin{document}

\thispagestyle{empty}

Name: \underline{\hspace{2in}} StudentID: \underline{\hspace{2in}} 26 Feb

This quiz will count toward your grade -- 1 point if you get the
correct answer.

Exact line search in 2 dimensions. For $w\in\mathbb R^2$, define the
cost function
$$C(w) = \frac 1 2 (w_1-1)^2 + \frac 1 2 (w_2+1)^2 = \frac 1 2 ||w +
\left[\begin{array}{c}
  -1\\
   1
\end{array}\right]
||^2_2$$

\vskip 1cm
Derive an expression for the gradient in terms of $w$, $\nabla C(w)=$\underline{\hspace{2in}}

Let $w^{(0)}=0$ be the starting point of gradient descent, at the
origin. 

\vskip 1cm
The descent direction is

$d^{(0)} = -\nabla C(w^{(0)})=$\underline{\hspace{2in}}

The cost of a step with size $\alpha>0$ in that direction is
\begin{equation*}
  \mathcal C_0(\alpha) = C(w^{(0)} + \alpha d^{(0)}).
\end{equation*}

To find the step size with the lowest cost we first need the derivative
(in terms of $\alpha$):

\vskip 1cm
\begin{equation*}
  \mathcal C'_0(\alpha) = \underline{\hspace{2in}}
\end{equation*}

\vskip 1cm 

Setting the derivative to zero, $\mathcal C'_0(\alpha)=0$,
then solving 

for $\alpha$
implies an optimal step size of
$\alpha^{(0)}=\argmin_\alpha \mathcal
C_0(\alpha)=$\underline{\hspace{2in}}

\vskip 1cm
Taking that step lands us at 

\begin{equation*}
  w^{(1)} = w^{(0)} + \alpha^{(0)} d^{(0)} = \underline{\hspace{2in}}
\end{equation*}

\vskip 1cm
which has a cost of

\begin{equation*}
  \mathcal C_0(\alpha^{(0)}) = C(w^{(1)}) = \underline{\hspace{2in}}
\end{equation*}


\end{document}
