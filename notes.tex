\documentclass{article}

\usepackage{algpseudocode}
\usepackage{amssymb}
\usepackage{amsmath}
\usepackage{fullpage}

\begin{document}

Each section is a new lecture.

\section{Applications}
\begin{itemize}
\item slides on applications of machine learning.
\item calculus/programming quiz.
\end{itemize}

\section{Training data in supervised learning}

Training data: $D= \{(x_1,y_1), \dots, (x_n,y_n)\}$, given/known.

Inputs $x_i\in\mathcal X$, outputs
$y_i\in\mathcal y$.

For each example, what is $\mathcal X,\mathcal Y$? 
\begin{itemize}
\item Image classification
\item Image segmntation
\item Translation
\item Spam filtering
\end{itemize}

Usually $\mathcal X = \mathbb R^p$ (e.g. word counts in emails).

Notation. 
\begin{itemize}
\item Inputs $x\in\mathbb R^p$.
\item $p$ = dimension of each input.
\item $n$ = number of training examples.
\item Outputs $y\in\mathbb R$ for regression, $y\in\{0,1\}$ for binary
  classification.
\item $f:\mathcal X\rightarrow \mathcal Y$ is the prediction function
  we want to learn.
\item Regression function $f:\mathbb R^p\rightarrow\mathbb R$
\item Binary classification function $f:\mathbb R^p\rightarrow \{0,1\}$.
\end{itemize}

Geometric interpretation of regression, height son/father (linear
pattern), species versus temperature (non-linear pattern).

Draw residual $r_i$ on graphs as vertical line segments.

\paragraph{Test data.} $D'= \{ (x'_1,y'_1), \dots, (x'_m,y'_m)\}$, unknown.
Goal is generalization to new/test data: algo $\textsc{Learn}(D) = f$,
with $f(x'_i)\approx y'_i$ for all test data, i.e. minimize
\begin{equation}
  \text{Err}_{D'}(f) = \sum_{(x',y')\in D'} \ell[f(x'), y']
\end{equation}
where $\ell$ is a loss function:
\begin{itemize}
\item In binary classification we typically use the
  mis-classification-rate/0-1-loss $\ell[f(x),y]=I[c_f(x)!=y]$, where
  the binary classifier $c_f(x)=0$ if $f(x)<0.5$ and 1 otherwise.
\item In regression we use the squared error $\ell[f(x),y]=[f(x)-y]^2$.
\end{itemize}
Since $D'$ is unknown, we need to assume that $D,D'\sim \mathcal P$.

\paragraph{Nearest neighbors algorithm.} What is near?
$\textsc{Dist}:\mathcal X\times \mathcal X \rightarrow \mathbb R_+$ is
a non-negative distance function between two data points.

e.g. for $\mathcal X=\mathbb R^p$ we use
\begin{itemize}
\item L1 distance $\textsc{Dist}(x,x') = ||x-x'||_1 = \sum_{j=1}^p |x_j-x_j'|$
\item L2 distance $\textsc{Dist}(x,x') = ||x-x'||_2 = \sqrt{
\sum_{j=1}^p (x_j-x_j')^2}$
\end{itemize}
Draw geometric interpretation of L1 and L2 distances.

Compute distances $d_1,\dots,d_n\in\mathbb R_+$ where 
$d_i=\textsc{Dist}(x_i,x)$.

Example graph, draw horizontal line segments $d_i$ between $x,x_i$.

Define neighbors function $N_{D,K}(x') = \{t_1, t_2, \dots, t_K\}$, where\\
$t_1,\dots,t_n\in\{1,\dots,n\}$ are indices such that\\
distances $d_{t_1}\leq \cdots \leq d_{t_n}$ are sorted ascending.

Predict the mean output value of the $K$ nearest
neighbors, $$f_{D,K}(x') = \frac 1 K \sum_{i\in N_{D,K}(x')} y_i$$

\begin{itemize}
\item For regression this is the mean.
\item For binary classification the mean is intepreted as a
  probability, predict 1 if greater than 0.5, and predict 0 otherwise.
\end{itemize}

\section{Nearest neighbors model selection and code}
\begin{itemize}
\item Review of supervised learning.
\item Definition of nearest neighbors prediction function $f_{D,K}(x)$.  
\item How to choose $K$? Discussion.
\item Formal definition of total error.
\item Visualizations. 1d regression and binary classification.
\item Draw train/validation matrices. Validation error.
\item Pseudocode for computing $f_{D,K}(x)$.
\end{itemize}

\begin{equation}
  \text{Verr}_{D,D'}(k) = \sum_{(x',y')\in D'} \ell[f_{D,k}(x'), y']
\end{equation}

\begin{algorithmic}[1]
  \State Function $\textsc{PredKNearestNeighbors}$
  \State Inputs: train inputs/features $x_1,\dots,x_n$, outputs/labels $y_1,\dots,y_n$,\\ \ \ \ test input/feature $x'$, max number of neighbors $K_{\text{max}}$:
  \For{$i=1$ to $n$}:
  \State $d_i\gets \textsc{Distance}(x', x_i)$
  \EndFor
  \State $t_1,\dots,t_n\gets \textsc{SortedIndices}(d_1,\dots,d_n)$
  \State $\text{totalY}\gets 0.0$
  \For{$k=1$ to $K_{\text{max}}$}:
  \State $i\gets t_k$
  \State $\text{totalY} \texttt{ += } y_i$
  \State Output: prediction $\text{totalY}/K_{\text{max}}$
\end{algorithmic}
Total complexity: $O(np + n\log n)$ where the Distance sub-routine is $O(p)$.

\section{Cross-validation and C++ code}
Remember we assume that training data and test data are randomly drawn
from a common probability distribution $\mathcal P$. To simulate that
process, we randomly assign a fold ID for each observation
$z_1,\dots,z_n\in\{1,\dots, \text{NumFolds}\}$. We then define
$\text{NumFolds}$ train/validation splits
\begin{equation}
  D_s = \{(x_i,y_i)|z_i = s\} \text{ --- validation set $s$}
  D_{-s} = \{(x_i,y_i)|z_i \neq s\} \text{ --- train set $s$}
\end{equation}
Draw train/validation/test matrices for $s=1$, $s=2$, etc.

Cross-validation data visualizations.

Mean validation error over all train/validation splits (folds):
\begin{equation}
  \text{MeanVerr}(k) = \text{Verr}_{D_{-s},D_{s}}(k)
\end{equation}

Overall we choose the parameter $\hat k = \argmin_k \text{MeanVerr}(k)$.

And the learning algorithm is $\textsc{LearnKNN}(D) = f_{D, \hat k}$.

How to code? First thing to do is write a sub-routine
\textsc{OptimalNeighborsCV} which computes the 

TODO: Geometric interpretation of binary classification in 1d,2d.

\end{document}
